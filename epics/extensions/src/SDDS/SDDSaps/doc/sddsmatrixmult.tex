% $Log: not supported by cvs2svn $
%
% Template for making SDDS Toolkit manual entries.
%
\begin{latexonly}
\newpage
\end{latexonly}

%
% Substitute the program name for <programName>
%
\subsection{sddsmatrixmult}
\label{sddsmatrixmult}

\begin{itemize}
\item {\bf description:}
%
% Insert text of description (typicall a paragraph) here.
%
\verb+sddsmatrixmult+ multiplies the matrices represented in the
two input files and puts the results in the output file.

String columns are ignored and not copied to the output file.

\item {\bf examples:} 
%
% Insert text of examples in this section.  Examples should be simple and
% should be preceeded by a brief description.  Wrap the commands for each
% example in the following construct:
% 
%
In an accelerator beamline a linear relationship exists between the
corrector dipole setpoints and the beam position monitor (BPM) readbacks.
The matrix data in file response is multiplied with the columns of 
file corrector to produce a new file containing values of expected bpm change:
\begin{flushleft}{\tt
sddsmatrixmult response correctorChange bpmExpectedChange
}\end{flushleft}
\item {\bf synopsis:} 
%
% Insert usage message here:
%
\begin{flushleft}{\tt
sddsmatrixmult [-pipe=[input][,output]] [{\em file1}] {\em file2} [{\em output}]
          [-commute] [-reuse] [-verbose] [-ascii]
}\end{flushleft}
\item {\bf files:}
% Describe the files that are used and produced
The first file ({\em file1}) is the SDDS file for left-hand matrix of product.
The second file ({\em file2}) is the SDDS file for right-hand matrix of product.
The third file contains the product matrix data.

\item {\bf switches:}
%
% Describe the switches that are available
%
    \begin{itemize}
    \item {\tt  -pipe[=input][,output]} --- The standard SDDS Toolkit pipe option.
    \item {\tt  -commute} --- Use {\em file1} for right-hand matrix and 
      {\em file2} for 
      left-hand matrix.  Useful with -pipe option
    \item {\tt  -reuse} --- If one file runs out of data pages, 
         then reuse the last one.
    \item {\tt  -ascii}  --- Produces an output in ascii mode. Default is binary.
    \item {\tt  -verbose} --- Write diagnostic messages to stderr.
    \end{itemize}
%\item {\bf see also:}
%    \begin{itemize}
%
% Insert references to other programs by duplicating this line and 
% replacing {\em prog} with the program to be referenced:
%
%    \item \progref{<prog>}
%    \end{itemize}
%
% Insert your name and affiliation after the '}'
%
\item {\bf author: L. Emery } ANL
\end{itemize}



