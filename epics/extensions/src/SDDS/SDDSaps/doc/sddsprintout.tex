\begin{latexonly}
\newpage
\end{latexonly}
\subsection{sddsprintout}
\label{sddsprintout}

\begin{itemize}
\item {\bf description:} \hspace*{1mm}\\
{\tt sddsprintout} provides formatted text output of data from columns and parameters.  It is similar to 
{\tt sdds2stream}, but provides better control of the appearance of the text.

\item {\bf examples:}
Make a printout of APS design beta functions along with the tunes:
\begin{flushleft}{\tt
sddsprintout APS0.twi -column=ElementName -column='beta?' -parameters='nu?' 
}\end{flushleft}

\item {\bf synopsis:} 
\begin{flushleft}{\tt
sddsprintout [-pipe[=input]] [{\em SDDSinput}] [{\em outputFile}] 
[-columns[={\em nameList}[,format={\em string}][,label={\em string}][,editLabel={\em command}]\\
 \hspace*{5mm} [,useDefaultFormat][,endsLine][,blankLines={\em number}]]]
[-parameters[={\em nameList}[,format={\em string}][,label={\em string}][,editLabel={\em command}] \\
 \hspace*{5mm} [,useDefaultFormat][,endsLine][,blankLines={\em number}]]]
[-spreadsheet[=delimiter={\em string}][,quoteMark={\em string}][,noLabels][,schFile={\em filename}]]
[-fromPage={\em number}] [-toPage={\em number}] 
[-formatDefaults={\em SDDStype}={\em formatString}[,...]]
[-width={\em integer}] [-pageAdvance] [-paginate[=lines={\em number}][,noTitle][,noLabels]]
[-postPageLines={\em number}] [-title={\em string}] [-noTitle] [-noWarnings]

}\end{flushleft}
\item {\bf files:}
{\em SDDSinput} is the SDDS file from which data is printed.  {\em outputFile} is a file to which the
printout will go; by default, the printout goes to the standard output.

\item {\bf switches:}
    \begin{itemize}
    \item {\tt -pipe[=input]} --- The standard SDDS Toolkit pipe option.
    \item {\tt -columns[={\em nameList}[,format={\em string}][,label={\em string}][,editLabel={\em command}]}
        {\tt [,useDefaultFormat][,endsline][,blankLines={\em number}]]} --- Specifies the names of 
        columns to appear in the printout.  {\em nameList} may contain one or more comma-separated
        strings, each of which may contain wildcards.  If more than one string is given, the list
        must be enclosed in parentheses, e.g., {\tt -columns='(betax,betay)'}.  

        The {\tt format} qualifier may be used to specify a {\tt printf}-style format string for the
        named columns; in this case, all of the columns must have the same data type.  The format
        string should contain a width field, to ensure proper alignment of text, e.g., \verb|%30s|
        rather than \verb|%s|. The {\tt useDefaultFormat} qualifier directs that {\tt sddsprintout}
        use its own default format for the data type in question, as opposed to any format that might
        be specified in the SDDS header.

        The {\tt label} qualifier can be used to specify the column label in the
        printout (by default, the column name is used); the label may be edited using the
        {\tt editLabel} qualifier and a standard editing sequence.

        If the {\tt endsLine} qualifier is given, a line break is issued after the last column of the
        list is printed.  The {\tt blankLies} qualifier may be used to specify that one or more
        blank lines be emitted following such a line break.

        Any number of {\tt -columns} options may be given.

    \item {\tt -parameters...} --- Identical to {\tt -columns}, except that printout of parameters
        is specified.


    \item {\tt -spreadsheet[=delimiter={\em string}][,quoteMark={\em string}][,noLabels][,schFile={\em filename}]} --- Specifies spreadsheet compatible output,
        using the given delimiter between columns. In this mode, simplified header is printed and
        no line width limits are imposed.  The default delimiter is a tab.  The default quotation mark
        is \".  If the {\tt schfile} qualifier is given, a header file for comma-separated-values
        data is generated. In this case, the delimiter should be a comma.

    \item {\tt fromPage={\em number}} --- Specifies the first data page of the file that will 
        appear in the printout.  By default, the printout starts with data page 1.

    \item {\tt toPage={\em number}} --- Specifies the last page of the file that will appear in 
        the printout.  By default, the printout ends with the last data page in the file.

    \item {\tt formatDefaults={\em SDDStype}={\em formatString}[,...]} --- Specifies default 
        {\tt printf} format strings for named SDDS data types.  The {\em SDDStype} qualifier may be
        one of {\tt float}, {\tt double}, {\tt long}, {\tt short}, {\tt string}, or {\em character}.

    \item {\tt -width={\em integer}} --- Specifies the width of the output line in number of characters. 
        The default is 130.

    \item {\tt -pageAdvance} --- Specifies that the page be advanced at the end of every data
        page of the SDDS file.  This is done by emitting any ASCII page advance character, which
        will probably work only if the output is sent to a printer.

    \item {\tt -paginate} --- Specifies pagination of the output, using a default 66 line page.
        The {\tt lines} qualifier may be used to change the page length.  By default, the title
        and column labels are printed for each page.  These may be disabled using the 
        {\tt notitle} and {\tt nolabels} qualifiers.

    \item {\tt -postPageLines} --- Specifies that a number of blank lines shall be emitted
        at the end of the printout for each page.  By default, there are no blank lines between
        pages.

    \item {\tt -title={\em string}} --- Specifies the title for the printout.
    \item {\tt -noTitle} --- Specifies that no title be printed.
    \item {\tt -noWarnings} --- Suppresses warning messages, such as those concerning data elements
        requested in the printout that are not in the input file.

    \end{itemize}
\item {\bf see also:}
        \begin{itemize}
        \item \hyperref{Data for Examples}{Data for Examples (see }{)}{exampleData}
        \item \progref{sdds2spreadsheet}
        \item \progref{sdds2stream}
        \end{itemize}
\item {\bf author:} M. Borland, ANL/APS.
\end{itemize}


