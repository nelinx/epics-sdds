\begin{latexonly}
\newpage
\end{latexonly}
\subsection{sddsconvolve}
\label{sddsconvolve}

\begin{itemize}
\item {\bf description:} 
{\tt sddsconvolve} performs discrete Fourier convolution/deconvolution/correlation of
signals in two files.  It assumes that spacing of points is the same in both input files.
\item {\bf example:}
Compute the result of a signal applied to a system with a known impulse response.
\begin{flushleft}{\tt
sddsconvolve signal.sdds impulseResponse.sdds signalResponse.sdds
-signalColumns=t,VSignal -responseColumns=t,VImpulse -outputColumns=t,VOutput
}\end{flushleft}

\item {\bf synopsis:}
\begin{flushleft}{\tt
sddsconvolve  {\em signal-file} {\em response-file} {\em output} [-pipe[=in][,out]]
 -signalColumns={\em indepColumn},{\em dataName}
 -responseColumns={\em indepColumn},{\em dataName}
 -outputColumns={\em indepColumn},{\em dataName}  
[{-deconvolve [-noiseFraction={\em value}] | -correlate}]
}\end{flushleft}
\item {\bf files:}
The meaning of the files depends on whether the {\tt -deconvolve} or {\tt -correlate}
options are given.
If neither option is given, then
{\em signal-file} is the file containing the signal that is imposed on the system,
{\em response-file} is the impulse response of the system, and 
{\em output} is the computed response of the system to the signal.
If {\tt -deconvolve} is given, then {\em signal-file} is the response of the
system to the signal, {\em response-file} is the impulse response of the system, and
{\em output} is the computed signal imposed on the system.
If {\tt -correlate} is given, then {\em signal-file} and {\em response-file} contain
two equivalent signals, while {\em output} contains the computed Fourier correlation;
physically, this tells over what time scale the two functions have correlated values.
\item {\bf switches:}
    \begin{itemize}
    \item {\tt -pipe=[input][,output] } --- The standard SDDS Toolkit pipe option.
        \item {\tt -signalColumns={\em indepColumn},{\em dataName}} --- Specifies the
        names of the data columns from {\em signal-file} (the first data file).
        \item {\tt -responseColumns={\em indepColumn},{\em dataName}} --- Specifies the
        names of the data columns from {\em response-file} (the second data file).
        \item {\tt -outputColumns={\em indepColumn},{\em dataName}} --- Specifies the
        desired names of the result in the file {\em output}.
        \item {\tt -deconvolve} --- Specifies deconvolution instead of convolution.
        \item {\tt -noiseFraction={\em value}} --- Specifies the amount of noise to
        allow in the deconvolution to prevent division by zero, as a fraction of the
        maximum power in the impulse response function.
        \item {\tt -correlate} --- Specifies correlation instead of convolution.
    \end{itemize}
\item {\bf author:} M. Borland, ANL/APS.
\end{itemize}


