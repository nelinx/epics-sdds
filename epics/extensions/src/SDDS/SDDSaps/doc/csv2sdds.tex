% $Log: not supported by cvs2svn $
% Revision 1.4  1998/08/13 16:11:43  borland
% Added documentation of -maxRows and -spanLines options.
%
% Revision 1.3  1998/07/16 15:07:07  borland
% Fixed error in -column syntax in example.
%
% Revision 1.2  1996/11/21 21:21:53  borland
% Changed several occurrences of CVS to CSV.
%
% Revision 1.1  1996/11/07 20:51:23  borland
% First version.
%
\begin{latexonly}
\newpage
\end{latexonly}
\subsection{csv2sdds}
\label{csv2sdds}

\begin{itemize}
\item {\bf description:} Converts Comma-Separated-Values (CSV) data and similar data to SDDS. CSV data
is commonly used by spreadsheet programs.
\item {\bf example:} 
\begin{flushleft}{\tt
csv2sdds data.csv -columnData=name=x,type=float,units=m -columnData=name=Name,type=string data.sdds
}\end{flushleft}
\item {\bf synopsis:}
\begin{flushleft}{\tt
csv2sdds [{\em CSVfile}] [{\em SDDSfile}] [-pipe[=in][,out]] 
[-asciiOutput] [-spanLines] [-maxRows={\em integer}]
[-schFile={\em SCHfilename}] [-skiplines={\em integer}]
[-delimiters=start={\em character},end={\em character}] [-separator={\em character}] 
[-columnData=name={\em string},type={\em string}[,units={\em string}] ...]
[-uselabels[=units]] [-majorOrder=<row|column>]
}\end{flushleft}
\item {\bf files: }
{\em CSVfile} is a comma-separated-values file.  Such a file consists of M rows each containing
N items of data, forming N columns.  The items on each row are separated by commas (or by a 
specified separator).  The items may also be delimited by double quotation marks (or by specified
delimiters).

{\em SDDSfile} is the SDDS output that is created.

The optional {\em SCHfilename} is a way of specifying the column headers.  The file is
expected to contain a series of lines of the form {\em tag}={\em valueList}, where {\em
valueList} is a comma-separated list of one or more items.  Lines not matching this format are
ignored.  The {\em tag} may be one of the following:
\begin{itemize}
\item {\tt Filetype}: optional.  If given, must have {\em valueList} of {\tt Delimited}.
\item {\tt Delimiter}: optional.  If given, the first character of {\em valueList} is used for
the start and end delimiters.
\item {\tt Separator}: optional.  If given, the first character of {\em valueList} is used for
the separator.
\item {\tt CharSet}: optional.  If given, must have {\em valueList} of {\tt ascii}.
\item {\tt Field}{\em N}, where {\em N} is an integer: one or more required.  The integers {\em N}
must be consecutive.  The first item in {\em valueList} is taken as the column name.  
The second item is interpreted as the data type.  At present, only the {\tt Float} data type is actually
interpreted as anything other than character data.  All others are
treated as character string types.  If needed, {\tt sddsprocess} may be used to process the
resulting string columns to produce other data types.
\end{itemize}

\item {\bf switches:}
    \begin{itemize}
    \item {\tt -pipe[=in][,out]} --- The standard SDDS Toolkit pipe option.
    \item {\tt -asciiOutput} --- Specifies ASCII output.
    \item {\tt -spanLines} --- Specifies that the program should ignore line breaks in parsing the 
        input data.
    \item {\tt -maxRows={\em integer}} --- The maximum number of rows expected.  This allows
        optimization of the program, but isn't essential.
    \item {\tt -schFile={\em filename} } --- Specifies the name of a SCH file specifying the
        format of the CSV file.  I don't know what SCH stands for, but apparently some PC
        programs generate such files.
    \item {\tt -skiplines={\em integer}} --- Skip the specified number of lines at the beginning 
        of the input file.
    \item {\tt -delimiters=start={\em character},end={\em character}} --- Specifies start and end
        delimiters for data.  The default is to use a double-quotation mark for both.
    \item {\tt -separator={\em character}} --- Specifies separator to use.  The default is a comma.
    \item {\tt -columnData=name={\em string},type={\em string}[,units={\em string}] } --- 
        Specifies the name and data
        type of a column of data in the CSV file.  One of these options should be given for each
        column in the input file, in the same order as the columns appear in that file. 
    \item {\tt -uselabels[=units]} --- The column names and optionally the units are defined in the
        file prior to the data.
    \item {\tt -majorOrder=<row|column>} --- Writes output file in row or column major order.
    \end{itemize}
\item {\bf see also:}
    \begin{itemize}
    \item \progref{sdds2stream}
    \item \progref{sddsprocess}
    \item \progref{plaindata2sdds}
    \end{itemize}
\item {\bf author:} M. Borland, ANL/APS.
\end{itemize}

