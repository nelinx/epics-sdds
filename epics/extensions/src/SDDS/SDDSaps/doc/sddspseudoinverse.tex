% $Log: not supported by cvs2svn $
% Revision 1.5  2000/09/16 00:46:41  emery
% Added documentation for
% options for weights file and u and v matrix printouts.
%
% Revision 1.4  1996/04/29  22:58:52  emery
% Minor spelling changes to make capitalization consistent.
%
% Revision 1.3  1996/04/28  22:58:10  emery
% Corrected the font of the option or file specification.
%
% Revision 1.2  1996/04/28  22:00:36  emery
% Added updated option and capabilities.
%
%
% Template for making SDDS Toolkit manual entries.
%
\begin{latexonly}
\newpage
\end{latexonly}

%
% Substitute the program name for <programName>
%
\subsection{sddspseudoinverse}
\label{sddspseudoinverse}

\begin{itemize}
\item {\bf description:}
%
% Insert text of description (typicall a paragraph) here.
%
\verb|sddspseudoinverse| views the numerical tabular data of the input file
as though it formed a matrix, and produces an output
file with data corresponding to the pseudo-inverse of the input
file matrix. At present the pseudo-inversion is done using
a singular value decomposition. Other methods may be made available in the future.

Command line options specifies the number of singular values to be
used in the inversion process.

The column names for the output file are generated either from the data in
a selected string column in the input file,
from the value of the command line option -root,
or from an internal default.

The column names of the input file are collected and made into
a string column in the output file.

\item {\bf examples:} 
%
% Insert text of examples in this section.  Examples should be simple and
% should be preceeded by a brief description.  Wrap the commands for each
% example in the following construct:
% 
%
The matrix of $R_{12}$ values for some accelerator beamline called LTP
is stored in file LTP.R12.
The pseudo-inverse (useful for trajectory correction), named
LTP.InvR12, is created with:

\begin{flushleft}{\tt
sddspseudoinverse LTP.R12 LTP.InvR12
}\end{flushleft}
\item {\bf synopsis:} 
%
% Insert usage message here:
%
\begin{flushleft}{\tt
sddspseudoinverse [<input>] [<output>] [-pipe=[input][,output]]
    [{-minimumSingularValueRatio=<value> | -largestSingularValues=<number>}] 
    [-smallestSingularValues=<number>] 
    [-deleteVectors=<list of vectors separated by comma>] 
    [-economy] [-printPackage] 
    [-oldColumnNames=<string>] [{-root=<string> [-digits=<integer>] | 
    -newColumnNames=<column>}] [-sFile=<file>[,matrix]] [-uMatrix=<file>] [-vMatrix=<file>] 
    [-weights=<file>,name=<columnname>,value=<columnname>]
    [-reconstruct=<file>] [-symbol=<string>] [-ascii] [-verbose] [-noWarnings]
    [-multiplyMatrix=<file>[,invert]]
}\end{flushleft}
\item {\bf files:}
% Describe the files that are used and produced
The input file contains the data for the matrix to be inverted. The output file
contains the data for the inverted matrix. If only one file is specified,
then the input file is overwritten by the output.

Multiple data pages of the input file will be processed and written to the
outptu file if all the data pages of the input file
have the same number of rows.
The processing will stop at the first data page
which doesn't have the same number of rows as that of the first page.
If applicable, the string column selected to generate column names for the
output file is assumed to be the same in all input data sets. The
string columns of only the first data set are read.
\item {\bf switches:}
%
% Describe the switches that are available
%
    \begin{itemize}
%
%   \item {\tt -pipe[=input][,output]} --- The standard SDDS Toolkit pipe option.
%
    \item {\tt  -pipe[=input][,output]} --- The standard SDDS Toolkit pipe option.
    \item {\tt  -minimumSingularValueRatio={\em realValue}} ---  Used to remove
        small singular values from the calculation. The smallest singular value retained
        for the inverse calculation
        is determined by multiplying this ratio value with the largest singular
        value of the input matrix.
    \item {\tt  -largestSingularValues={\em integer}} ---  Used to remove
        small singular values from the calculation. The largest 
        {\tt {\em integer}} singular values are kept.
    \item {\tt -deleteVectors}  -deleteVectors=n1,n2,n3,... which will set the inverse singular values 
                  of modes n1,n2,n3, ect to zero. 
               The order in which the SV removal options are processed is 
               minimumSingularValueRatio, largestSingularValues and then deleteVectors.
    \item {\tt -economy} --- If given,  only the first min(m,n) columns for the U matrix are calculated or returned 
               where m is the number of rows and n is the number of columns. This 
               can potentially reduce the computation time with no loss of useful information.
               economy option is highly recommended for most pratical applications since it uses
               less memory and runs faster. If economy option is not give, a full m by m U matrix 
               will be internally computated no matter whether -uMatrix is provided. 
    \item {\tt  -oldColumnNames={\em string}} --- 
        A string column  of name {\tt {\em string}} is created in the output file, containing
        the column names of the input files as string data.
        If this option is not present, then the default name of ``OldColumnNames''
        is used for the string column.
    \item {\tt -multiplyMatrix={\em file}[,invert]} --- if invert is not provided,  then the output matrix is the inverse of the input
               matrix multiplying by this matrix; otherwise, the output matrix is the product of 
               multiply matrix and the inverse of the input matrix.
    \item {\tt  -root={\em string}} ---
        A string used to generate columns names for the output file data. 
        The first data column is named ``{\tt {\em string}000}'',
        the second, ``{\tt {\em string}001}'', etc.
    \item {\tt  -digits={\em integer}} --- minimum number of digits used in the number 
        appended to {\tt {\em root}} of the output file column names. (Default value is 3).
    \item {\tt -sFile={\em file}} --- writes the singular values vector to file.
    \item {\tt  -newColumnNames={\em string}} --- Specifies a string column
        of the input file which will be used to define column names
        of the output file.
    \item {\tt  -umatrix={\em file}} --- writes the $u$ column-orthogonal matrix 
       to a file. The SVD decomposition follows the convention 
       $A = u S v^T$. The ``transformed'' $x$ are $v^T x$, and 
        the ``transformed'' $y$ are $u^T y$. 
    \item {\tt  -vmatrix={\em file}} --- writes the $v$ column-orthogonal matrix to a file.
    \item {\tt  -removeDCVectors} --- Removes the eigenvectors which have an overall DC component.
    \item {\tt  -weights={\em file},name={\em columnName},value={\em columnName}]} 
        --- Specifies a file which contains weights for each of 
        the rows of the matrix, thus giving different weights for solving the 
        linear equations of the pseudoinverse problem.
        The equation that is solved is $wAx = wy$ where $w$ is the weight vector
        turned into a diagonal matrix and $A$ is the input matrix.
        The matrix solution returned is $(wA)^I w$ where $()^I$ means taking 
        the pseudoinverse. The u matrix now has a different interpretation:
        the ``transformed'' $x$ are $v^T x$, as before, but the 
         ``transformed'' $y$ are $u^T w y$.
    \item {\tt  -symbol={\em string}} --- The string for the symbol 
   field of data column definitions.
    \item {\tt -reconstruct} --- speficy a file which will reconstruct the original matrix with only the
               singular values retained in the inversion.
    \item {\tt -printPackage} --- prints out the linear algebra package that was compiled.
    \item {\tt  -ascii}  --- Produces an output in ascii mode. Default is binary.
    \item {\tt  -verbose} --- Prints out incidental information to stderr.
    \end{itemize}
%\item {\bf see also:}
%    \begin{itemize}
%
% Insert references to other programs by duplicating this line and 
% replacing <prog> with the program to be referenced:
%
%    \item \progref{<prog>}
%    \end{itemize}
%
% Insert your name and affiliation after the '}'
%
\item {\bf author: L. Emery } ANL
\end{itemize}

