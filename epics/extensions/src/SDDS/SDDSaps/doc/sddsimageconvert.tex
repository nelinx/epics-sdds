% $Log: not supported by cvs2svn $
%
%
\begin{latexonly}
\newpage
\end{latexonly}
\subsection{sddsimageconvert}
\label{sddsimageconvert}

\begin{itemize}
\item {\bf description:} Converts a single-column SDDS image file into a multi-column SDDS image file and vice versa.
\item {\bf example:} 
\begin{flushleft}{\tt
sddsimageconvert image.input image.output 
}\end{flushleft}
\item {\bf synopsis:}
\begin{flushleft}{\tt
sddsimageconvert [{\em Inputfile}] [{\em Outputfile}] [-pipe[=in][,out]] 
[-ascii] [-binary]
[-multicolumn=[indexName={\em name}][,prefix={\em name}]]
[-singlecolumn=[imageColumn={\em name}][,xVariableName={\em name}][,yVariableName={\em name}]]
[-nowarnings]
}\end{flushleft}
\item {\bf files: }

{\em Inputfile} is the SDDS input image file. This file can be either a single-column image file or a multi-column image file.

{\em Outputfile} is a single-column SDDS image file if the {\em Inputfile} is a multi-column SDDS image file, or it is a multi-column SDDS image file if the {\em Inputfile} is a single-column SDDS image file.

\item {\bf switches:}
    \begin{itemize}
    \item {\tt -pipe[=in][,out]} --- The standard SDDS Toolkit pipe option.
    \item {\tt -ascii} --- The {\em Outputfile} is written in ascii format.
    \item {\tt -binary} --- The {\em Outputfile} is written in binary format.
    \item {\tt -multicolumn=[indexName={\em name}][,prefix={\em name}]} --- The multi-column SDDS image file will have or does have an index column with the name given by indexName={\em name}. The default name is Index. It also will have or does have multiple columns with the prefix given by prefix={\em name}. For an input file this defaults to the prefix of the first column found that is not the index column and that ends with a number. For an output file this defaults to the same name as the image column name in the single-column SDDS image file.
    \item {\tt -singlecolumn=[imageColumn={\em name}][,xVariableName={\em name}][,yVariableName={\em name}]} --- The single-column SDDS image file will have or does have an image column with the name given by imageColumn={\em name}. The default is the name of only column that exists in the single-column input file or the image prefix in the multi-column input file. If the output file is a single-column image file the xVariableName={\em name} and yVariableName={\em name} options will be used to define the x and y variable names. These default to x and y.
    \item {\tt -nowarnings} --- No warnings will be issued when the input file is overwritten.
    \end{itemize}
\item {\bf see also:}
    \begin{itemize}
    \item \progref{sddscongen}
    \item \progref{sddscontour}
    \item \progref{sddsimageprofiles}
    \item \progref{sddsspotanalysis}
    \end{itemize}
\item {\bf author:} R. Soliday, ANL/APS.
\end{itemize}

