\begin{latexonly}
\newpage
\end{latexonly}
\subsection{sddsenvelope}
\label{sddsenvelope}

\begin{itemize}
\item {\bf description:}
{\tt sddsenvelope} analyzes column data across pages to find minima,
maxima, averages, standard-deviations, etc., on a row-by-row basis.
It produces a single-page output file containing one column for each analysis
requested.  It will also copy through data from the first page into the output file.
It requires that each page of the input file have the same number of rows.
\item {\bf examples:}
Find the minimum and maximum beta functions for a set of APS lattices:
\begin{flushleft}{\tt
sddsenvelope APS.twi APS.twi.env -copy=s -minimum=beta? -maximum=beta?
}\end{flushleft}
\item {\bf synopsis:}
\begin{flushleft}{\tt
sddsenvelope [-pipe=[input][,output]] [{\em input}] [{\em output}] [-copy={\em columnNames}] 
[-maximum={\em columnNames}] [-minimum={\em columnNames}] 
[-mean={\em columnNames}] [-sum={\em power},{\em columnNames}] 
[-standardDeviation={\em columnNames}] [-rms={\em columnNames}] 
[-slope={\em independentVariableName},{\em columnNames}]
[-intercept={\em independentVariableName},{\em columnNames}] 
[-median={\em columnNames}] [-decileRange={\em columnNames}]
}\end{flushleft}
\item {\bf files:}
      {\em inputFile} is a multipage file containing the data for which row-by-row 
      statistics are desired.  {\em outputFile} is a single-page file containing the    
      statistics.  The column names in {\em outputFile} are created from those in the
      input file by appending the appropriate suffix from the following list:
      {\tt Max}, {\tt Min}, {\tt Mean}, {\tt StDev}, {\tt RMS}, {\tt Sum}, {\tt Slope}, or
      {\tt Intercept}.
\item {\bf switches:}
    \begin{itemize}
    \item {\tt -pipe=[input][,output]} --- The standard SDDS Toolkit pipe option.
    \item {\tt -copy={\em columnNames}} --- Specifies that the named columns should be transferred
        to the output file without alteration.  These data come from the first page
        of the input file.  A comma-separated list of optionally wildcard-containing
        strings may be given.
    \item {\tt -maximum={\em columnNames}, -minimum={\em columnNames}}, 
        {\tt -mean={\em columnNames}}, {\tt -rms={\em columnNames}},
        {\tt -median={\em columnNames}}, {\tt -decileRange={\em columnNames}} ---
        Specifies that the named columns should be analysed in the indicated fashion.
        A comma-separated list of optionally wildcard-containing strings may be given.
        Decile range is the spread between the 90\% and 10\% points on the distribution.
    \item {\tt -sum={\em power},{\em columnNames}} ---Specifies that the named columns 
        should be analysed in the indicated fashion,
        i.e., that each output row should be the sum of the values to the indicated power.
        A comma-separated list of optionally wildcard-containing strings may be given.
    \item {\tt -slope={\em independentVariableName},{\em columnNames},
         -intercept={\em independentVariableName},{\em columnNames}}--- Specifies that the named columns should be analysed 
        to get the slope or intercept with respect to the parameter {\em independentVariableName}.
        A comma-separated list of optionally wildcard-containing strings may be given for
        the {\em columnNames}.
    \end{itemize}
\item {\bf see also:}
    \begin{itemize}
    \item \hyperref{Data for Examples}{Data for Examples (see }{)}{exampleData}
    \item \progref{sddschanges}
    \end{itemize}
\item {\bf author:} M. Borland, ANL/APS.
\end{itemize}

