\begin{latexonly}
\newpage
\end{latexonly}

\subsection{sddsdigfilter}
\label{sddsdigfilter}

\begin{itemize}
\item {\bf description:}

{\tt sddsdigfilter} performs time-domain digital filtering of columns of data. Filters can
be combined in series and/or cascade to produce complex filter characteristics. In
addition to allowing simple 1-pole lowpass and highpass filters, filter charateristics can
be defined using either digital 'Z' or analog 'S' domain transfer functions.

A digital filter has a Z transform given by
\[
\frac{b_0 + b_1 z^{-1} + \ldots + b_n z^{-n}}{a_0 + a_1 z^{-1} + \ldots + a_n z^{-n}},
\]
while an analog filter has a Laplace transform given by
\[
\frac{d_0 + d_1 s^{1} + \ldots + d_n s^{n}}{c_0 + c_1 s^{1} + \ldots + c_n s^{n}},
\]

\item {\bf examples:} 
These examples assume the existence of a file {\tt data.wf} containing a waveform
stored as a column {\tt value} that is a function of a column {\tt time} that has
units of seconds.

 Pass data through lowpass filter with a -3dB cutoff of 0.01 Hz:

  \begin{flushleft}{\tt
  sddsdigfilter data.wf -col=time,value result.wf -low=1,0.01.
  }\end{flushleft}

  Bandstop filter between 10 Hz and 100 Hz:

  \begin{flushleft}{\tt
  sddsdigfilter data.wf -col=time,value result.wf -low=1,10 -high=1,100
  }\end{flushleft}

  Bandpass filter between 10 Hz and 100 Hz:

  \begin{flushleft}{\tt
  sddsdigfilter data.wf -col=time,value result.wf -low=1,100 -cascade -high=1,10
  }\end{flushleft}

  Analog transfer function:

  \begin{flushleft}{\tt
  sddsdigfilter data.wf -col=time,value result.wf -analog=D,1.0,0.01,C,0.1,0.3,1.6
  }\end{flushleft}

  Five-sample digital delay:

  \begin{flushleft}{\tt
  sddsdigfilter data.wf -col=time,value result.wf -digital=B,0,0,0,0,0,1
  }\end{flushleft}
 
\item {\bf synopsis:} 
\begin{flushleft}{\tt
sddsdigfilter [{\em inputFile}] [{\em outputFile}] [-pipe=[input][,output]]
  -columns={\em xName},{\em yName}
 [-proportional={\em gain}]
 [-lowpass={\em gain},{\em cutoffFrequency}]
 [-highpass={\em gain},{\em cutoffFrequency}]
 [-digitalfilter={\em sddsfile},{\em aCoeffName},{\em bCoeffName}
 [-digitalfilter=[A,{\em a0},{\em a1},..,{\em am}][,B,{\em b0},{\em b1},..,{\em bn}]
 [-analogfilter={\em sddsfile},{\em cCoeffName},{\em dCoeffName}
 [-analogfilter=[C,{\em c0},{\em c1},..,{\em cm}][,D,{\em d0},{\em d1},..,{\em dn}]
 [-cascade]
 [-verbose]

}\end{flushleft}
\item {\bf files:}
 Two file names are required: the name of the existing input file,
 and the name of the output file to be produced. The input file must
 contain at least two columns: one containing to data to be filtered
 ({\em yName}) and the other giving time information ({\em xName}). A linear time
 scale is assumed for {\em xName}.
 The output file is a copy of the input file with an additional column
 called {\tt DigFiltered{\em yName}} where {\em yName} would be the name of the
 original y-column.

\item {\bf switches:}
%
% Describe the switches that are available
%
    \begin{itemize}

   \item {\tt -pipe[=input][,output]} --- The standard SDDS Toolkit pipe switch.
   \item {\tt -columns={\em xName},{\em yName}} --- The names of the input file data columns.
   \item {\tt -proportional={\em gain}} --- Defines a gain stage, where {\em gain} is the multiplier applied to the data.

   \item {\tt -lowpass={\em gain},{\em cutoffFrequency}} --- Defines a lowpass filter stage, where
{\em gain} is the mutiplier applied to the data and {\em cutoffFrequency} is the -3dB point of the
filter in units appropriate to the supplied {\em xName}.

   \item {\tt -highpass={\em gain},{\em cutoffFrequency}} --- Defines a highpass filter stage,
where {\em gain} is the multiplier applied to the data and {\em cutoffFrequency} is the -3dB point
of the filter in units appropriate to the supplied {\em xName}.

   \item {\tt -digitalfilter={\em sddsfile},{\em aCoeffName},{\em bCoeffName}} --- Defines a digital
filter with coefficients in the supplied SDDS coefficient file. This file must cointain two
columns containing the A and B coefficients of a digital 'Z' transfer function. Note that
control theory convention assumes that the A0 coefficient is always 1.0. To ensure
consistency with the SDDS file, the a0 coefficient is the first row in the A-column and
must be implicitly supplied. Although there is little benefit to setting a0
to anything other than 1.0, it is allowed.

 \item {\tt -digitalfilter=[A,{\em a0},{\em a1},...,{\em am}][,B,{\em b0},{\em b1},...,{\em bn}]} --- Defines a
digital filter with the A and B coefficients of the digital 'Z' transfer function supplied
on the command line. Either A or B or both coefficients can be supplied. If no A
coefficients are supplied, a0 is set to 1.0. Equally, if no B coefficients are supplied,
b0 is set to 1.0. If different numbers of A and B coefficients are suppied, the filter
order is determined from the largest order.

   \item {\tt -analogfilter={\em sddsfile},{\em cCoeffName},{\em dCoeffName}} --- Defines an analog
filter with coefficients in the supplied sdds cefficient file. This file must cointain two
columns containing the C and D coefficients of an analog 's' transfer function. Conversion
to the digital domain is done using a bilinear transform. Note that the user must ensure
adequate data sampled, since the general format does not allow frequency warping based on
the filter cutoff frequency.

 \item {\tt -analogfilter=[A,{\em a0},{\em a1},...,{\em am}][,B,{\em b0},{\em b1},...,{\em bn}]} --- Defines an
analog filter with the C and D coefficients of the analog 'S' transfer function supplied
on the command line. Either C or D or both coefficients can be supplied. If no C
coefficients are supplied, then c0 is set to 1.0. Equally, if no D coefficients are
supplied, then d0 is set to 1.0. Conversion to the digital domain is done using a bilinear
transform. Note that the user must ensure adequate data sampled, since the general format
does not allow frequency warping based on the filter cutoff frequency.

   \item {\tt -cascade} --- Defines the start of a new filter stage. Any number of filter
stages can be supplied for a single data set. If more than one filter is defined, then the
outputs are summed unless the {\tt -cascade} switch is supplied between the filter definitions
in which case the output of the first filter stage is fed into the input of the subsequent
filter stage.

   \item {\tt -verbose} --- Prints the filter coefficients for each filter stage.
   \end{itemize}
\item {\bf references} --- 
	The digital filtering routines were adapted from Stearns and David, {\em Signal Processing Algorithms in Fortran and C}, Prentice Hall, 1993

\item {\bf author}: John Carwardine, Argonne National Laboratory
\end{itemize}

