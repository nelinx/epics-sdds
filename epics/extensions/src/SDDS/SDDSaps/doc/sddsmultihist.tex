\begin{latexonly}
\newpage
\end{latexonly}
\subsection{sddsmultihist}
\label{sddsmultihist}

\begin{itemize}
\item {\bf description:} 
{\tt sddsmultihist} does one-dimensional histograms of multiple columns of data from an SDDS file.
All columns are histogrammed on the same  interval and with the same number of bins.
It is similar to {\tt sddshist}, except that the latter program only histograms a single
column at a time.  Unlike {\tt sddshist}, {\tt sddsmultihist} does not presently do
statistical analyses or filtering.

\item {\bf examples:} 
Make 20-bin histogram of a group of PAR x beam-position-monitor readouts:
\begin{flushleft}{\tt
sddshist par.bpm par.bpmhis -column=P?P?x -bins=20 -abscissa=xReadout
}\end{flushleft}
\item {\bf synopsis:} 
\begin{flushleft}{\tt
sddsmultihist [-pipe=[input][,output]] [{\em inputFile}] [{\em outputFile}]
-columns={\em columnName}[,{\em columnName}...] -abscissa={\em newName} 
[-separate]
[-exclude={\em columnName}[,{\em columnName}...]]
[{-bins={\em integer} | -sizeOfBins={\em value} | -autobins=target={\em number}[,minimum={\em integer}][,maximum={\em integer}]}]
[-lowerLimit={\em value}] [-upperLimit={\em value}] 
[-sides] 
}\end{flushleft}
\item {\bf files:} {\em inputFile} is the name of an SDDS file
containing data to be histogrammed. If {\em inputFile} contains
multiple data pages, each is treated separately.  The histograms are
placed in {\em outputFile}, which has one column of histogram
frequencies for each histogrammed input column, plus a column giving
the abscissa values for the frequency distributions.  The former
columns have names of the form {\em columnName}{\tt Frequency},
containing the number of points in each bin.  The latter column has a
name given by the user.

\item {\bf switches:}
    \begin{itemize}
    \item \verb|-pipe[=input][,output]| --- The standard SDDS Toolkit pipe option.
    \item {\tt -columns={\em columnName}[,{\em columnName}...]} --- Specifies the names
         of the data columns to be histogrammed.
        The {\em columnName} items may contain wildcards.
    \item {\tt -separate} --- Specifies that a separate abscissa shall be created for each histogrammed
        column.  If {\tt -abscissa} is not given, then the abscissa names are the names of the columns
        being histogrammed. 
    \item {\tt -abscissa={\em newName}[,{\em newName}...]} --- Specifies the name or names of the 
        abscissa columns for the histogram output.  If {\tt -separate} is not given, then only
        one name is permitted.
        The units taken from the units of the columns being histogrammed.
    \item {\tt -exclude={\em columnName}[,{\em columnName}...]} --- Specifies the names of data columns to
        exclude from histogramming.  The {\em columnName} items may contain wildcards.
    \item {\tt -bins={\em number}} --- Specifies the number of bins to use.  The default is 20.
    \item {\tt -sizeOfBins={\em value}} --- Specifies the size of bins to use.  The number of bins is
        computed from the range of the data.
    \item {\tt -autoBins=target={\em number}[,minimum={\em integer}][,maximum={\em integer}]} --- Specifies
      that the number of bins should be chosen to attempt to give a target number of samples per bin on average.
      If {\tt minimum} is given, then no fewer than the specified number of bins will be used (default: 5).
      If {\tt maximum} is given, then no more than the specified number of bins will be used (default: number of samples).
    \item {\tt -lowerLimit={\em value}} --- Specifies the lower limit of the histogram.  By default,
        the lower limit is the minimum value in the data.
    \item {\tt -upperLimit={\em value}} --- Specifies the upper limit of the histogram.  By default,
        the upper limit is the maximum value in the data.
    \item {\tt -sides} --- Specifies that zero-height bins should be attached to the lower
        and upper ends of the histogram.  Many prefer the way this looks on a graph.
    \end{itemize}
\item {\bf see also:}
    \begin{itemize}
    \item \hyperref{Data for Examples}{Data for Examples (see }{)}{exampleData}
    \item \progref{sddshist}
    \item \progref{sddshist2d}
    \end{itemize}
\item {\bf author:} M. Borland, ANL/APS.
\end{itemize}

