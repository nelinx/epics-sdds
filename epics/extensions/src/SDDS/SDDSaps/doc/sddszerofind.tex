\begin{latexonly}
\newpage
\end{latexonly}
\subsection{sddszerofind}
\label{sddszerofind}

\begin{itemize}
\item {\bf description:}

{\tt sddszerofind} finds the locations of zeroes in a single column of an SDDS file.  This is done by finding successive rows
for which a sign change occurs in the ``dependent column'', or any row for which an exact zero is present in this column.  For
each of the ``independent columns'', the location of the zero is determined by linear interpolation.  Hence, the program
is really interpolating multiple columns at locations of zeros in a single column.  This single column is in a sense
being looked at as a function of each of the interpolated columns.

\item {\bf examples:} 
Find zeroes of a Bessel function, ${\rm J_0(z)}$:
\begin{flushleft}{\tt
sddszerofind J0.sdds J0.zero -zero=J0 -column=z
}\end{flushleft}
Find zeroes of a Bessel function, ${\rm J_0(z)}$, and simultaneously interpolate ${\rm J_1(z)}$ at
the zero locations:
\begin{flushleft}{\tt
sddszerofind J0.sdds J0.zero -zero=J0 -column=z,J1
}\end{flushleft}
(This isn't the most accurate way to interpolate ${\rm J_1(z)}$, of course.)

\item {\bf synopsis:} 
\begin{flushleft}{\tt
sddszerofind [-pipe=[input][,output]] [{\em inputfile}] [{\em outputfile}] 
-zeroesOf={\em columnName} [-columns={\em columnNames}] [-slopeOutput]
}\end{flushleft}
\item {\bf files:}
{\em inputFile} contains the data to be searched for zeroes.  {\em outputFile} contains columns for each of the independent
quantities and a column for the dependent quantity.  Normally, each dependent quantity is represented by a single column of
the same name.  If output of slopes is requested, additional columns will be present, having names of the form 
{\tt {\em columnName}Slope}.  

If {\em inputFile} contains multiple pages, each is treated separately and is delivered to a separate page of {\em
outputFile}.

\item {\bf switches:}
    \begin{itemize}
    \item \verb|-pipe[=input][,output]| --- The standard SDDS Toolkit pipe option.
    \item {\tt -zeroesOf={\em columnName}} --- Specifies the name of the dependent quantity, for which zeroes
        will be found.
    \item {\tt -columns={\em columnNames}} --- Specifies the names of the independent quantities, for which
        zero locations will be interpolated.  Generally, there is only one of these.  {\em columnNames}
        is a comma-separated list of optionally wildcarded names.
    \item {\tt -slopeOutput} --- Specifies that additional columns will be created containing the slopes of
        the dependent quantity as a function of each independent quantity.  This can be useful, for example,
        if one wants to pick out only positive-going zero-crossings.
    \end{itemize}
\item {\bf see also:}
    \begin{itemize}
    \item \progref{sddsinterp}
    \end{itemize}
\item {\bf author:} M. Borland, ANL/APS.
\end{itemize}

