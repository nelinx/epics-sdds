\begin{latexonly}
\newpage
\end{latexonly}
\subsection{sddsdistest}
\label{sddsdistest}

\begin{itemize}
\item {\bf description:} 
{\tt sddsdistest} performs the Kolmogorov-Smirnov (K-S) test on a
set of numbers to determine how likely those numbers are to have been drawn from
a specified statistical distribution (e.g., gaussian, poisson).

\item {\bf example:}
Try the K-S test on random numbers generated by {\tt sddsprocess}
\begin{flushleft}{\tt
sddssequence -pipe=out -define=i,type=long -sequence=begin=0,end=9999,delta=1 
| sddsprocess -pipe -define=column,gaussRN,grnd -define=column,uniformRN,rnd 
| sddsdistest -pipe -test=ks -gaussian -column=gaussRN -column=uniformRN 
| sddsprintout -pipe -column=ColumnName -column=distestSigLevel
}\end{flushleft}
The result is 
\begin{flushleft}{\tt
    ColumnName      distestSigLevel 
-------------------------------------
     gaussRN         4.019061e-01 
    uniformRN        1.598565e-32 
}\end{flushleft}
which shows that the K-S test accurately distinguishes between numbers
drawn from the two distributions.  The probability that the numbers in
column {\tt uniformRN} are from a gaussian distribution is very small, whereas
the probability that the numbers in column {\tt gaussRN} are from a
gaussian distribution is 40\%.
\item {\bf synopsis:}
\begin{flushleft}{\tt
sddsdistest [{\em input}] [{\em output}] [-pipe=[in][,out]] 
-column={\em name}[,sigma={\em name}] ... -exclude={\em name}[,{\em name}...] ...
{-gaussian | -poisson | -student | -chisquared }
[-degreesOfFreedom={{\em value} | @{\em parameterName}}]
}\end{flushleft}
\item {\bf switches:}
    \begin{itemize}
    \item {\tt -pipe=[input][,output] } --- The standard SDDS Toolkit pipe option.
        \item {\tt -column={\em name}[,sigma={\em name}]} --- Specifies the name of
        a column to test, and optionally the name of the column with the measurement
        error for the each test value.  {\em name} may contain wildcards.  The sigma
        name may contain ``\%s'', for which each column name is substituted to obtain
        the corresponding sigma name.  Multiple {\tt column} options may be given.
        \item {\tt -exclude={\em name}[,{\em name}...]} --- Specifies the names of
        columns to exclude from testing.
        \item {\tt -gaussian | -poisson | -student | -chisquared } --- Specifies the
        model distribution against which to test the data.
        \item {\tt -degreesOfFreedom={{\em value} | @{\em parameterName}}} ---
        Specifies the number of degrees of freedoms to assume for the model distribution
        in the case of student and chi-squared distribution.  The first form specifies a
        fixed {\em value}, whereas the second specifies taking the value for each page
        from the named parameter.
    \end{itemize}
\item {\bf author:} M. Borland, ANL/APS.
\end{itemize}


