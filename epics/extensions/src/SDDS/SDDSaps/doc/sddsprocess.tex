\begin{latexonly}
\newpage
\end{latexonly}
\subsection{sddsprocess}
\label{sddsprocess}

\begin{itemize}
\item {\bf description:} \hspace*{1mm}\\
\verb|sddsprocess| operates on the data columns and parameters of an existing SDDS data set and creates a new
data set.  The program supports filtering and matching operations on both tabular data and parameter
data, definition of new parameters and columns in terms of existing ones, units conversions, scanning of 
string data to produce numeric data, composition of string data from other data types, 
statistical and waveform analyses, and other operations.
\item {\bf examples:}
Compute the square-roots of the beta-functions, which are the beam-size envelopes:
\begin{flushleft}{\tt
sddsprocess APS.twi -define=column,sqrtBetax,"betax sqrt"  -define=column,sqrtBetay,"betay sqrt" 
}\end{flushleft}
Compute the horizontal beam-size, given by the equation
\[ \sigma_x = \sqrt{ \epsilon_x \beta_x + (\eta_x \sigma_\delta)^2} \]
\begin{flushleft}{\tt
sddsprocess APS.twi -define=parameter,epsx,8.2e-9,units=nm -define=parameter,sigmaDelta,1e-3 
  -define=column,sigmax,"epsx betax *  sigmaDelta etax * sqr + sqrt",units=m
}\end{flushleft}
\item {\bf synopsis:} 
\begin{flushleft}{\tt
sddsprocess [-pipe[=input][,output]] [{\em inputFile}] [{\em outputFile}] {\em options}
}\end{flushleft}
\item {\bf files:}
{\em inputFile} is an SDDS file containing data to be processed.  If no options are given, it is copied to {\em
outputFile} without change.  {\em Warning:} if no output filename is given, and if an output pipe is not selected,
then the input file will be replaced.

\item {\bf switches:}
    \begin{itemize}
    \item Data winnowing: Any number of the following may be used.  They are applied in the order
given.  Note that {\tt -match} and {\tt -test} are the most time intensive; thus, if several types
of winnowing are to be applied, these should be used last if possible.

   \begin{itemize}

   \item {\tt -filter=\{column | parameter\},{\em rangeSpec}[,{\em rangeSpec}[,{\em logicOp}...]] } ---
Specifies winnowing {\em inputFile} based on numerical data in parameters or columns.  A {\em range-spec} is of the
form {\tt {\em name},{\em lower-value},{\em upper-value}[,!] }, where \verb|!| signifies logical negation. A page
passes a given filter by having the named parameter inside (or outside, if negation is given) the specified range,
where the endpoints are considered inside.  A tabular data row passes a given filter in the analogous fashion, except
that the value from the named column is used.  

One or more range specifications may be combined to give a accept/reject status by employing the {\em
logic-operations}, \verb|&| (logical and) and \verb&|& (logical or).  For example, to select rows for
which A is on [0, 1] and B is on [10, 20], one would use
{\tt -filter=column,A,0,1,B,10,20,\&}.
        \item {\tt -timeFilter=\{column | parameter\},[before=YYYY/MM/DD@HH:MM:SS] [,after=YYYY/MM/DD@HH:MM:SS][,invert]} 
--- Specifies date range in YYY/MM/DD@HH:MM:SS format in time parameters or columns. The invert option cause the
filter to be inverted, so that the data that would otherwise be kept is removed and vice-versa. For example,
if one want to keep data between 8:30AM on Januaray 2, 2003 and 9:20PM on February 6,2003, the option woould be
     -timeFilter=column,Time,before=2003/2/6@21:20,after=2003/1/2@8:30
assume that the time data is in the column Time.

        \item {\tt -match=\{column | parameter\},{\em matchTest}[,{\em matchTest},{\em logicOp}]} --- Specifies
winnowing {\em inputFile} based on data in string parameters or columns.  A {\em match-test} is of the form {\tt {\em
name}={\em matchingString}[,!]}, where the matching string may include the wildcards \verb|*| (matches zero of more
characters) and \verb|?| (matches any one character). 

If the first character of {\em matchingString} is '@', then the remainder of the string is taken to be the name of a
parameter or column.  In this case, the match is performed to the data in the named entity.  For column-based matching,
this is done row-by-row.  For parameter-based matching, it is done page-by-page.

In addition, if instead of \verb|=| one uses \verb|=+|, then matching is case-insensitive.  The plus sign is intended
to be mnemonic, as the case-insensitive matching results in additional matches.

The use of several match tests and logic is done just as for {\tt -filter}.  For example, to match
all the rows for which the column {\tt Name} starts with 'A' or 'B', one could use
{\tt -match=column,Name=A*,Name=B*,|}.  (This could also be done with {\tt -match=column,Name=[AB]*}.)

\item {\tt -numberTest=\{column | parameter\},{\em name}[,invert]} --- 
Specifies testing the values of in a
string column (parameter) to see if they can be (or cannot be, if {\tt
invert} is given) converted to numbers.  If not, the corresponding row
(page) is deleted.  

\item {\tt -test=\{column | parameter\},{\em test}[,autostop][,algebraic]} --- Specifies
winnowing of {\em inputFile} based on a test embodied in an {\tt rpn} expression.  The expression, {\em test}, may use
the names of any parameters or columns.  If {\tt autostop} is specified, the processing of the data set (or data page)
terminates when the parameter-based (or column-based) expression is false.

        \item {\tt -clip={\em head},{\em tail}[,invert]} --- Specifies the number of data points to clip from the head
and tail of each page.  If {\tt invert} is given, the clipping retains rather than deletes the indicated points.
         \item {\tt -fclip={\em head},{\em tail}[,invert]} --- Specifies the fraction of data points to clip from the head
and tail of each page.  If {\tt invert} is given, the clipping retains rather than deletes the indicated points.
        \item {\tt -sparse={\em interval}[,{\em offset}]} --- Specifies sparsing of each page with the indicated
interval.  That is, only every {\em interval}$ {th}$ row starting with row {\em offset} is copied to the
output.  The default value of {\em offset} is 0.

        \item {\tt -sample={\em fraction}} --- Specifies random sampling of rows such that approximately the indicated
fraction is kept.  Since a random number generator is used that is seeded with the system clock, this will usually
never be the same twice.

        \end{itemize}
    \item \verb|rpn| calculator initialization:\\

        \begin{itemize} \item {\tt -rpnDefinitionsFiles={\em filename}...} --- Specifies a list of comma-separated
filenames to be read in as \verb|rpn| definitions files.  By default, the file named in the {\tt RPN\_DEFNS}
environment variable is read.

        \item {\tt -rpnExpression={\em expression}[,repeat][,algebraic]} --- Specifies an \verb|rpn| expression to be executed.  If
{\tt repeat} is not specified, then the expression is executed before processing begins.  If {\tt repeat} is specified,
the expression is executed just after each page is read; it may use values of any of the numerical parameters for that
page. This option may be given any number of times.

        \end{itemize}
  \item Scanning from, editing, printing to, and executing string columns and parameters: \\

         \begin{itemize}
 
         \item {\tt -scan=\{column | parameter\},{\em newName},{\em sourceName},{\em sscanfString}}\\ {\tt [,{\em
 definitionEntries}]} --- Specifies creation of a new numeric column (parameter) by scanning an existing string column
 (parameter) using a \verb|sscanf| format string.  The default type of the new data is double; this may be changed by
 including a {\em definitionEntry} of the form {\tt type=}{\em typeName}.  With the exception of the {\tt name} field,
 any valid namelist command field and value may be given as part of the {\em definitionEntries}.

        If {\em sourceName} contains wildcards, then {\em newName} must contain at least one occurrence of the string
``\%s''.  In this case, for each name that matches {\em sourceName}, an additional element is created, with a
name created by substituting the name for ``\%s'' in {\em newName}.
 
         \item {\tt -edit=\{column | parameter\},{\em newName},{\em sourceName},{\em edit-command}} --- Specifies
creation of a new string column (parameter) called {\em newName} by editing an existing string column (parameter) {\em
sourceName} using an emacs-like editing string.  For details on editing commands, see \hyperref{SDDS editing}{SDDS
editing (see }{)}{SDDS editing}.
         
        If {\em sourceName} contains wildcards, then {\em newName} must contain at least one occurrence of the string
``\%s''.  In this case, for each name that matches {\em sourceName}, an additional element is created, with a
name created by substituting the name for ``\%s'' in {\em newName}.

         \item {\tt -reedit=\{column | parameter\},{\em name},{\em edit-command}} --- Like {\tt -edit},
except that the element {\em name} must already exist.  Each value is replaced by the value obtained from
applying {\em edit-command}.
         
        \item {\tt -print=\{column | parameter\},{\em newName},{\em sprintfString},{\em sourceName}}\\ {\tt [,{\em
sourceName}...][,{\em definitionEntries}]} --- Specifies creation of a new string column (parameter) by formatted
printing of one or more elements from other columns (parameters).  The {\em sprintfString} is a C-style format string
such as might be given to the routine {\tt sprintf}.  With the exception of the {\tt name} field, any valid namelist
command field and value may be given as part of the {\em definitionEntries}.

        \item {\tt -reprint} --- Identical in syntax and function to {\tt -print}, except that if {\em newName} 
already exists, it is overwritten.  No error or warning is issued.

        \item {\tt -format=\{column | parameter\},{\em newName},{\em
sourceName}} \\ {\tt [,stringFormat={\em sprintfString}][,doubleFormat={\em
sprintfString}]} \\ {\tt [,longFormat={\em sprintfString}]} --- Reformats string
data in different ways depending on the type of data the string
contains.  Each string is separated into tokens at space boundaries.
Each token is separately formatted, either as a long integer, a
double-precision floating point number, or a string, depending on what
the token appears to be.  The formatting is done using the specified
format strings; the default format strings are \%ld for longs,
\%21.15e for doubles, and \%s for strings.

         \item {\tt -system=\{column | parameter\},{\em newName},{\em commandName},}\\ {\tt [{\em
definitionEntries}]} --- Specifies creation of a new string column (parameter) by executing an existing string
column (parameter) using a subprocess.  The first line of output from the subprocess is acquired and placed in the
new column (parameter).

        If {\em commandName} contains wildcards, then {\em newName} must contain at least one occurrence of
``\%s''.   In this case, for each name that matches {\em commandName}, an additional element is created, with a
name created by substituting the name for ``\%s'' in {\em newName}.

 
         \end{itemize}
 
    \item Creation and modification of numeric columns and parameters: \\

        \begin{itemize}

        \item {\tt -convertUnits=\{column | parameter\},{\em
name},{\em newUnits},}{\tt {\em oldUnits},{\em factor}}\\ ---
Specifies units conversion for the column or parameter {\em name}
(which may contain wildcards).  The {\em factor} entry the factor by
which the values must be multiplied to convert them to the desired
units.  It is an error if {\em oldUnits} does not match the original
units of the column or parameter.  Eventually, the {\em factor} entry
will be made optional by inclusion of conversion information in the
program.  This option may be given any number of times.

        \item {\tt -define=\{column | parameter\},{\em name},{\em
equation}[,select={\em matchString}][,exclude={\em matchString}]} 
{\tt [,editSelection={\em editCommand}][,{\em definitionEntries}][,algebraic] }\\ --- Specifies
creation of a new column or parameter using an {\tt rpn} expression to
obtain the values.  For parameters, any parameter value may be
obtained by giving the parameter name in the expression.  For columns,
one may additionally get the value of any column by giving its name in
the expression; the expression given for {\tt -define=column} is
essentially specifying a vector operation on columns with parameters
as scalars.  By default, the type of the new data is \verb|double|.
This and other properties of the new column or parameter may be
altered by giving {\em definitionEntries}, which have the form {\em
fieldName}={\em value}; {\em fieldName} is the name of any namelist
command field (except the name field) for a column or parameter, as
appropriate.  This option may be given any number of times.

        Using the {\tt select} qualifier, it is possible to use a
single {\tt -define} option to specify many instances of new column
definitions.  If {\tt select} is given, the input is searched for all
the column names matching {\em matchString}.  These are then
optionally editted using the {\em editCommand} specified with {\tt
editSelection}.  The resulting strings are then substituted one at a
time into {\em name} and {\em equation}, replacing all occurances of
``\%s''.  For example, suppose a file contained a number of
column-pairs of the form {\em Prefix}V1 and {\em Prefix}V2; to take
the difference of each pair, one could use\\ {\tt
-define=column,\%sDiff,\%sV1 \%sV2 -,select=*V1,edit=\%/V1//} \\[12pt]
{\tt sddsprocess} permits read access to individual elements of a
column of data using the {\tt rpn} array feature.  For each column, an
array of name \verb|&|{\em ColumnName} is created; the ampersand is to
remind the user that the variable \verb|&|{\em ColumnName} is the
address of the start of the array.  To get the first element of a
column named \verb|Data|, one would use {\tt 0 \&Data [}.  This will
function only within or following a {\tt -define=column} or {\tt
-redefine=column} operation.  It is an error to attempt to access data
beyond the bounds of an array.

   The number of columns, and the current page and row number are 
pre-loaded into the rpn calculator memory according to the following table.
\begin{center}
\begin{tabular}{|l|l|} \hline
Quantity    & rpn memory   \\ \hline
Page number & {\tt i\_page}       \\ \hline
Page number & {\tt table\_number} \\ \hline
Row number  & {\tt i\_row}        \\ \hline
Number of rows & {\tt n\_rows}    \\ \hline
\end{tabular}
\end{center}
For example, to generate a column of index number to a file, add the option {\tt
-define=col,Index,i\_row,type=long}.

        \item {\tt -redefine} --- This option is identical to {\tt
-define} except that the column or parameter already exists in the
input.  The equation may use the previous values of the entity being
redefined by including the column name in the expression.

        \item {\tt -evaluate=\{column | parameter\},{\em name},{\em
source}[,{\em definitionEntries}]}\\ --- Specifies creation of a new column or parameter {\em name}
containing values from evaluation of the equation stored in a string column or parameter {\em source}.
The source string is an rpn expression in terms of the other column and parameter values.

        \item {\tt -cast=\{column | parameter\},{\em newName},{\em oldName},{\em newType}} --- This option allows
casting data from one numerical data type to another.  It is much faster than trying to do the same operation using
{\tt -define}.  The string {\em newType} may be any of {\tt double}, {\tt float}, {\tt long}, {\tt short}, or
{\tt character}.

        \item {\tt -process={\em mainColumnName},{\em analysisName},{\em resultName}}
\\{\tt [,description={\em string}][,symbol={\em string}][,weightBy={\em columnName}]}
\\{\tt [,functionOf={\em columnName}[,lowerLimit={\em value}][,upperLimit={\em value}]]}
\\{\tt [,head={\em number}][,tail={\em number}][,fhead={\em fraction}][,ftail={\em fraction}]}
\\{\tt [,topLimit={\em value}][,bottomLimit={\em value}]} 
\\{\tt [,position][,offset={\em value}][,factor={\em value}]}
\\{\tt [,match={\em columnName},value={\em match-value}]} --- This option may be given
any number of times.  It specifies creation of a new parameter {\em resultName} by processing
column {\em mainColumnName} using analysis mode {\em analysisName}.  The column must contain
numeric data, in general, except for a few analysis modes that take any type of data (see below). {\em
mainColumnName} may contain wildcards, in which case the processing is applied to all matching
columns containing numeric data.  {\em resultName} may have a single occurence of the string
``\%s'' embedded in it; if so, {\em mainColumnName} is substituted.  If wildcards are given in
{\em mainColumnName}, then ``\%s'' must appear in {\em resultName}; in this case, the name of
each selected column is substituted.  Similarly, if the {\tt description} field is supplied,
it may contain an embedded ``\%s'' for which the column name will be substituted.

Recognized values for {\em analysisName} are:

\begin{itemize}
\item {\tt average}, {\tt rms}, {\tt sum}, {\tt standardDeviation}, {\tt mad} --- The arithmetic average, 
the rms average, the arithmetic sum, the standard deviation, and the mean absolute deviation.  All may be possibly
weighted.
\item {\tt median}, {\tt drange}, {\tt qrange} --- The median value, i.e., the value which is both above and below
50\% of the data points; the decile-range, which is the range excluding the smallest and largest 10\% of the values;
the quartile-range, which is the range excluding the smallest and largest 25\% of the values.
\item {\tt percentile}, {\tt prange} --- These compute percentiles and percentile ranges, as defined by the 
  {\tt percentlevel} qualifier.  For {\tt percentile}, the value returned is the value of the column corresponding
  to the given {\tt percentlevel}.  For {\tt prange}, the value return is the span of the values in the column
  encompassing the given central percentage of the data; for example {\tt percentile=50} would give the quartile range.

\item {\tt minimum}, {\tt maximum}, {\tt spread}, {\tt smallest}, {\tt largest} --- 
The minimum value, maximum value, spread in values, smallest value (minimum absolute value), and largest value
(maximum absolute value).  For all except {\tt spread}, the {\tt position} and {\tt functionOf} qualifiers may be
given to obtain the value in another column when {\em mainColumnName} has the extremal value; the {\tt functionOf} qualifer may name a string column.

\item {\tt first}, {\tt last} --- The values in the first and last rows of the page.  Will accept non-numeric
data.
\item {\tt pick} --- The first value within the filter.  Will accept non-numeric data.
\item {\tt count} ---  The number of values in the page.
\item {\tt baselevel}, {\tt toplevel}, {\tt amplitude} --- Waveform analysis parameters from histogramming
the signal amplitude.  {\tt baselevel} is the baseline, {\tt toplevel} is the height, and {\tt amplitude}
is height above baseline.

\item {\tt risetime}, {\tt falltime}, {\tt center} --- The rise and fall times from the 10\%-90\% and
90\%-10\% transitions.  {\tt center} is the midpoint between the first 50\% rising edge and the first following 50\%
falling edge after rising above 90\% amplitude.  Requires specifying a independent variable column with {\tt
functionOf}.

\item {\tt fwhm}, {\tt fwtm}, {\tt fwha}, {\tt fwta} --- Full-widths of the named column as a function
of the independent variable column specified with {\tt functionOf}.  The letters 'h' and 't' specify Half and Tenth
amplitude widths, while 'm' and 'a' specify Maximum value or Amplitude over baseline.

\item {\tt zerocrossing} --- Zero-crossing point of the column named with {\tt functionOf} of the
column {\em mainColumnName}.

\item {\tt sigma} --- The standard deviation over the square-root of the number of points.  This is an
estimate of the uncertainty in the mean value.

\item {\tt slope}, {\tt intercept}, {\tt lfsd} --- The slope and intercept of a linear fit.  The {\tt functionOf}
qualifier must be given to specify the quantity to fit against.  {\tt lfsd} is the Linear-Fit-Standard-Deviation,
which is the standard deviation of the fit residuals.

\item {\tt gmintegral} --- The integral of the quantity with respect to the quantity named with the {\tt functionOf} qualifier.
  The integral is performed using the Gill-Miller method, which works well for non-equispaced values of the independent
  variable.

\item {\tt correlation} --- The Pearson's correlation coefficient of the quantity and the column of which it is 
  (nominally) a function (as declared with the {\tt functionOf} qualifier).

\end{itemize}

Qualifiers for this switch are:
\begin{itemize}
\item {\tt description={\em string}}, {\tt symbol={\em string}} --- Specify the description and symbol
fields for the new column.
\item {\tt weightBy={\em columnName}} --- Specifies the name of a column to weight values from 
column {\em mainColumnName} by before computing statistics.
\item {\tt functionOf={\em columnName}} --- Specifies the name of a column that {\em mainColumnName}
is to be considered a function of for computing widths, zero-crossings, etc.
\item {\tt topLimit={\em value}}, {\tt bottomLimit={\em value}} --- Specifies winnowing of rows
so that only those with {\em mainColumn} values above the {\tt topLimit} or below the {\tt bottomLimit} are 
included in the computations.
\item {\tt lowerLimit={\em value}}, {\tt upperLimit={\em value}} --- If {\tt functionOf} is given,
specifies winnowing of rows so that only rows for which the independent column data is above
the {\tt lowerLimit} and/or below the {\tt upperLimit} are included in computations.
\item {\tt head={\em number}}, {\tt fhead={\em fraction}} --- Specifies taking the head 
of the data prior to processing.  {\tt head} gives the number of points keep, while {\tt fhead}
gives the fraction of the points to keep.  If {\em number} or {\em fraction} is less than 0, thenthe head points are deleted and the other points are kepts. If head and tail are both used, head  is
performed first.
gives the fraction of the points to clip.
\item {\tt tail={\em number}}, {\tt ftail={\em fraction}} --- Specifies taking the tail 
of the data prior to processing.  {\tt tail} gives the number of points keep, while {\tt ftail}
gives the fraction of the points to keep.  If {\em number} or {\em fraction} is less than 0, thenthe tail points are deleted and the other points are kepts. If head and tail are both used, head  is
performed first.
\item {\tt position} --- For {\tt minimum}, {\tt maximum}, {\tt smallest}, and {\tt largest} analysis modes,
specifies that the results should be the position at which the indicated value occurs.  This position is
the corresponding value of in column named with {\tt functionOf}.
\item {\tt offset={\em value}}, {\tt factor={\em value}} --- Specify an offset and factor for modifying
data prior to processing.  By default, the offset is zero and the factor is 1.  The equation is
$x \rightarrow f*(x+o) $.
\item {\tt match={\em controlName}}, {\tt value={\em match-value}} --- Specify the match column and the
match value (may contain wildcard).
\end{itemize}


\end{itemize}


    \item Miscellaneous:
        \begin{itemize}
        \item {\tt -ifis=\{column | parameter | array\},{\em name}[,{\em name}...]}\\ {\tt -ifnot=\{column | parameter |
array\},{\em name}[,{\em name}...]} \\ --- These options allow conditional execution.  If any column that is named under a
\verb|ifis| option is not present, execution aborts.  If any column that is named under a \verb|ifnot| option is
present, execution aborts.
        \item {\tt -description=[text={\em string}][,contents={\em string}]} --- Specifies the
        description fields for the SDDS dataset.  Use of this feature is disparaged as these fields
        are not manipulated by any tools.  Use of string parameters is suggested.
        \item {\tt -summarize} --- Specifies that a summary of the processing be printed to the screen.
        \item {\tt -verbose} --- Specifies that informational printouts be provided during processing.
        \item {\tt -noWarnings} --- Specifies suppression of warning messages.
    \item {\tt -delete=\{columns | parameters | arrays\},{\em matchingString}[,...]},\\
        {\tt -retain=\{columns | parameters | arrays\},{\em matchingString}[,...]}
         --- These options specify wildcard strings to be used to select entities
        (i.e., columns, parameters, or arrays) that will respectively be deleted or retained (i.e., that will not or
        will appear in the output).   
        The selection is performed by determining which input entities have names matching any of the strings.
        If \verb|retain| is given but \verb|delete| is not, only those entities matching one of the
        strings given with \verb|retain| are retained.  If both \verb|delete| and \verb|retain|
        are given, then all entities are retained except those that match a \verb|delete| string without
        matching any of the \verb|retain| strings.
        \end{itemize}
    \end{itemize}

\item {\bf author:} M. Borland, H. Shang, R. Soliday ANL/APS.
\end{itemize}

