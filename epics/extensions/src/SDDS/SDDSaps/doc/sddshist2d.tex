\begin{latexonly}
\newpage
\end{latexonly}

\subsection{sddshist2d}
\label{sddshist2d}

\begin{itemize}
\item {\bf description:}

{\tt sddshist2d} makes two-dimensional histograms of data, producing
output that is suitable for plotting with {\tt sddscontour}.
The two-dimensional histogram may include data from two columns, or may
show the histograms of a single column versus page number.

\item {\bf examples:} 
Make a two-dimensional histogram of two PAR bpm readouts, then plot the result:
\begin{flushleft}{\tt 
sddshist2d par.bpm par.bpm.h2d -column=P1P1x,P1P2x -xparam=50 -yparam=50
sddscontour -shade=32 par.bpm.h2d -quantity=frequency
}
\end{flushleft}

\item {\bf synopsis:} 
\begin{flushleft}{\tt
sddshist2d [-pipe[=input][,output]] [{\em inputfile}] [{\em outputfile}]
-columns=\{{\em xName},{\em yName} | {\em yName}\}
[-weights={\em columnName}[,average]]
[-xParameters={\em bins}[,{\em lower},{\em upper}]] [-yParameters={\em bins}[,{\em lower},{\em upper}]]
[-outputName={\em string}] 
[-sameScale] [-combine] [-normalize[=sum]] [-smooth[={\em passes}]] 
[-verbose] 
}\end{flushleft}

\item {\bf switches:}
    \begin{itemize}
    \item {\tt -pipe[=input][,output]} --- The standard SDDS Toolkit pipe option.
    \item {\tt -columns=\{{\em xName},{\em yName} | {\em yName}\}} --- Specifies the data from
        the input to histogram.  If both {\em xName} and {\em yName} are given, then {\tt sddshist2d}
        does a two-dimensional histogram of the values in the named columns.  If only {\tt yName}
        is given, then {\tt sddshist2d} does a series of one-dimensional histograms of the named
        column, one for each data pages; these histograms are then assembled as a two-dimensional
        histogram with one axis being the page number.
    \item {\tt -weights={\em columnName}[,average]} --- Specifies the name of a column of data with
        which to proportionally weight the count value of points in the histogram.  If the {\tt average}
        qualifier is given, then each bin value is normalized to contain the average value of the weight for
        all points in the bin.
    \item {\tt -xParameters={\em bins}[,{\em lower},{\em upper}]} --- Specifies the number of bins and
        optionally the histogrammed region for the x values.  Ignored if only {\em yName} is given.
        By default, 21 bins are used encompassing all of the data points.
    \item {\tt -yParameters={\em bins}[,{\em lower},{\em upper}]} --- Specifies the number of bins and
        optionally the histogrammed region for the y values.  By default, 21 bins are used encompassing
        all of the data points.
    \item {\tt -outputName={\em string}} --- Specifies the name of the histogram data.  The default
        is {\tt frequency}.
    \item {\tt -sameScale} --- Specifies that for multipage input files, the histogram region should be
        the same for all pages.  The region is set to encompass all data points from all pages.
    \item {\tt -combine} --- Specifies that for multipage input files, the data from all pages should be
        placed in a single histogram.
    \item {\tt -normalize[=sum]} --- Specifies normalization of the histogram.  If the {\tt sum} qualifier
        is not given,  the histogram is normalized to unit amplitude; otherwise, it is normalized so that
        the sum of all frequencies is unity.
    \item {\tt -smooth[={\em passes}]} --- Specifies smoothing by nearest-neighbor-averaging.  If {\em passes}
        is omitted, only one pass is performed.
    \item {\tt -verbose} --- Requests informational output during processing.
    \end{itemize}
\item {\bf see also:}
    \begin{itemize}
    \item \progref{sddshist}
    \item \progref{sddscontour}
    \item \progref{sddscongen}
    \end{itemize}
\item {\bf author:} M. Borland, ANL/APS.
\end{itemize}
