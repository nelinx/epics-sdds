\begin{latexonly}
\newpage
\end{latexonly}
\subsection{sddsnaff}
\label{sddsnaff}

\begin{itemize}
\item {\bf description:}
{\tt sddsnaff} is an implementation of Laskar's Numerical Analysis of Fundamental
Frequencies (NAFF) algorithm.  This algorithm provides a way of determining the frequency
components of a signal that is more accurate than Fast Fourier Transforms (FFT). 
FFTs are used as part of the analysis, so if an FFT is sufficient for an application,
{\tt sddsfft} should be used as it will be much faster.

The algorithm starts by removing the average value of the signal and
applying a Hanning window.  Next, the signal is FFT'd and the
frequency at which the maximum FFT amplitude occurs is found.  This is
taken as the starting frequency for a numerical optimization of the
``overlap'' between the signal and $e^{i\omega t}$, which allows
determining $\omega$ to resolution greater than the frequency spacing
of the FFT.  Once $\omega$ is determined, the overlap is subtracted
from the original signal and the process is repeated, if desired.

\item {\bf examples:} 
Find the first fundamental frequency for each of the BPM signals in {\tt par.bpm}.
\begin{flushleft}{\tt
sddsnaff par.bpm par.naff -column=Time,'P?P?x' -terminateSearch=frequencies=1
}\end{flushleft}
\item {\bf synopsis:} 
\begin{flushleft}{\tt
sddsnaff [{\em inputfile}] [{\em outputfile}]
[-pipe=[input][,output]]
[-columns={\em indep-variable}[,{\em depen-quantity}[,...]]] 
[-pair={\em<column1>,<column2>}]
[-exclude={\em depen-quantity}[,...]]
[-terminateSearch={changeLimit={\em fraction}[,maxFrequencies={\em number}] | frequencies={\em number}}]
[-iterateFrequency=[cycleLimit={\em number}][,accuracyLimit={\em fraction}]]
[-truncate] [-noWarnings]
}\end{flushleft}
\item {\bf files:}

{\em inputFile} contains the data to be NAFF'd.  One column from this
file must be chosen as the independent variable.  If {\em inputFile}
contains multiple pages, each is treated separately and is delivered
to a separate page of {\em outputFile}.

{\em outputFile} contains two columns for each selected column in {\em
inputFile}.  These columns have names like {\em origColumn}{\tt
Frequency} and {\em origColumn}{\tt Amplitude}, giving the frequency
and amplitude for {\em origColumn}.

\item {\bf switches:}
    \begin{itemize}
    \item \verb|pipe[=input][,output]| --- The standard SDDS Toolkit pipe option.
    \item {\tt -columns={\em indepVariable}[,{\em depenQuantityList}]} --- Specifies the name of the
        independent variable column.  Optionally if no -pair options given, specifies a list of comma-separated, optionally
        wildcard-containing names of dependent quantities to be NAFF'd as a function of the 
        independent variable. 
        By default, all numerical columns except the independent column are NAFF'd.
     \item {\tt -pair={\em <column1>,<column2>}]} --- Specifies the names of the conjugate pairs to
        give double the frequency range. {\em<column1>} is used to obtain the basic frequency, 
        {\em<column2>} is used to obtain the phase at the frequency of the first column. 
        The relative phase between them will expand the resulting frequency from 0 - Fn to 0 - 2*Fn.
        Multiple -pair options may be provided. The indepentdent 
        column is provided by -columns option. The dependent columns may be provided by either -columns
        or -pair option. 
    \item {\tt -exclude={\em depenQuantity},...} --- Specifies optionally wildcarded names of columns
        to exclude from analysis.
    \item {\tt -terminateSearch=\{changeLimit={\em fraction}[maxFrequencies={\em number}] | }
        {\tt frequencies={\em number}\}} --- Specifies when to stop searching for frequency 
        components.  If {\tt changeLimit} is given, then the program stops when the RMS change
        in the signal is less than the specified {\em fraction} of the original RMS value of
        the signal.  The maximum number of frequencies that will be returned in this mode
        is specified with {\tt maxFrequencies} (default is 4).  If {\tt frequencies} is given,
        then the program finds the given number of frequencies, if possible.
        By default, the program finds one frequency for each signal.
    \item {\tt -iterateFrequency=[cycleLimit={\em number}][,accuracyLimit={\em fraction}]} ---
        This option controls the optimization procedure that searches for the best frequency.
        By default, the procedure executes 100 passes and attempts to determine the frequency
        to a precision of 0.00001 of the Nyquist frequency.  
        {\tt cycleLimit} is used to change the number of passes, while {\tt accuracyLimit} is used
        to specify the desired precision.
    \item \verb|-truncate| --- Specifies that the data should be truncated so that the number of points is 
        the largest product of primes from 2 to 19 not greater than the original number of points.  
        In some cases, this will result in significantly greater speed, by making the FFTs faster.
    \item \verb|-noWarnings| --- Suppresses warning messages.
    \end{itemize}
\item {\bf see also:}
    \begin{itemize}
    \item \hyperref{Data for Examples}{Data for Examples (see }{)}{exampleData}
    \item \progref{sddsfft}
    \end{itemize}
\item {\bf author:} M. Borland, ANL/APS.
\end{itemize}

